\section{Benchmark Specifications}

%%%%%%%%%%%%%%%%%%%%%%%%%%%%%%%%%%%%%%%%%%%%%%%%%%%%%%%%%%%%%%%%%%%%%%%%%%%%%%%%
\subsection{Overview}

The core geometry specifications are described in 3 levels of increasing scope,
detailing each of the hierarchical elements of the model. First the radial
geometry is described, followed by a section detailing the axial parameters.

At the lowest level, the radial geometry of each of the pincell types used
throughout the core is described. Next, the fuel assembly design is detailed, 
including the possible configurations of burnable absorbers and the radial
specification of the grid spacers. Finally, the greater core geometry is
described, including the fuel assembly enrichment locations, the positions of
burnable absorbers, instrument tubes, control rod banks, and shutdown banks, as
well as the baffle that surrounds the fuel assemblies, the core barrel, four
neutron shield panels, and the reactor pressure vessel and liner.

After the radial and axial geometry descriptions, a material specifications
section lists the details of each of the materials referred to.
%Finally, the last section details the approximations made by this model and
%suggests addition approximations that might be appropriate.
Table \ref{overview_table} provides a summary of key model parameters that will
be specified in greater detail in subsequent sections, and Figure
\ref{fig_overview} shows a core cross section indicating the radial structures
and assembly loading pattern.
\newpage
\begin{table}
  \centering
  \caption{Summary of key model parameters. \label{overview_table}}
  \begin{tabularx}{\textwidth}{l C c}
    \toprule
    \multicolumn{3}{c}{\phantom{Source}\hfill Core Lattice \hfill Source}\\
    \midrule
    \midrule
    & & \\
    
    No. Fuel Assemblies & 193 & \ref{num:assycore}\\
    & & \\
    Loading Pattern & w/o U235 & \\
    ~~Region 1 (cycle 1) & 1.60$^\dagger$ & \ref{num:assycore}\\
    ~~Region 2 (cycle 1) & 2.40$^\dagger$ & \ref{num:assycore}\\
    ~~Region 3 (cycle 1) & 3.10$^\dagger$ & \ref{num:assycore}\\
    ~~Region 4A (cycle 2) & 3.20$^{\dagger\dagger}$ & \ref{num:assycore}\\
    ~~Region 4B (cycle 2) & 3.40$^{\dagger\dagger}$ & \ref{num:assycore}\\
    & & \\
    
    Cycle 1 Heavy Metal Loading & 81.8 MT & \ref{num:assyload} \\
    & & \\
    \midrule 
    \multicolumn{3}{c}{Fuel Assemblies}\\
    \midrule
    \midrule
    & & \\
    
    Pin Lattice Configuration & $17 \times 17$ & \ref{num:fuellattice}\\
    Active Fuel Length & 365.76 cm & \ref{num:fuelheight}\\
    No. Fuel Rods & 264 & \ref{num:fuellattice}\\
    No. Grid Spacers & 8 & \ref{num:fuellattice}\\
    
    & & \\
    \midrule
    \multicolumn{3}{c}{Control}\\
    \midrule
    \midrule
    & & \\
    
    Control Rod Material (Upper Region) & \hyperlink{mat_b4c_rod}{B4C} & \ref{num:b4c_rod_mat}\\
    Control Rod Material (Lower Region) & \hyperlink{mat_aic_rod}{Ag-In-Cd} & \ref{num:aic_rod_mat}\\
    No. Control Rod Banks & 53 & \ref{num:assycore}\\
    
    & & \\
    No. Burnable Poison Rods in Core & 1266 & \ref{num:assycore}\\
    Burnable Poison Material & \hyperlink{mat_borosilicate}{Borosilicate Glass}, 12.5 w/o $\mathrm{B}_2\mathrm{O}_3$ & \ref{num:fuellattice}\\
    
    & & \\
    \midrule
    \multicolumn{3}{c}{Performance}\\
    \midrule
    \midrule
    & & \\
    
    Core Power & 3411 MWth & \ref{num:measurement}\\
    Operating Pressure & 2250 psia & \ref{num:measurement}\\
    Core Flow Rate & $61.5\times10^6\,\mathrm{kg/hr}$ (5\% bypass$^\ddagger$) & \ref{num:flowrate} \\
    \bottomrule 
  \end{tabularx}
  \begin{small}\begin{flushleft}
    \noindent $^\dagger$ Cycle 1 Actual core-averaged enrichments calculated from detailed assembly loadings, see Source \ref{num:assy_load}. \\
    \noindent $^{\dagger\dagger}$ Cycle 2 Actual core-averaged enrichments calculated from detailed assembly loadings, see Source \ref{num:assy_load_c2}. \\
    \noindent $^\ddagger$ It is assumed that 5\% of core flow rate goes core into bypass region. A fraction of this flow rate passes through
                          guide tubes. The flow rate should be estimated so that no boiling occurs in these regions.
  \end{flushleft}\end{small}
\end{table}
\clearpage
\begin{figure}[hptb]
  \centering
  
  \begin{tikzpicture}[x=1cm, y=1cm]

    \def\sqrttwo{1.4142135624}
  
    % actual dimensions in cm
    \def\coreBarrelOR{193.6750}
    \def\rpvOR{241.300}
    \def\rpvIR{219.710}
    \def\shieldOR{201.630}
    \def\baffleOR{163.4998}  % lattice pitch * 7.5 + baffle width  (21.50364*7.5+2.2225)

    \def\figwidth{10.16} % figure width in cm, assumes figure is cropped to rpvOR
  
    % all the dimensions set based on fig being inculded as 4in
    \draw[white] (-6,-6) rectangle (6,6);
    \node {\pgftext{\includegraphics[width=\figwidth cm]{specifications/figs/pwr_core.png}}};
    \draw[red,thick,->]  (-2*2.54,2*2.54) node[left, anchor=south] {Pressure Vessel} --  (-\figwidth/2/\sqrttwo,\figwidth/2/\sqrttwo);
    \draw[red,thick,->]  (2*2.54,2*2.54) node[right, anchor=south] {Neutron Shield Panel} --  (\shieldOR*\figwidth/2/\rpvOR/\sqrttwo,\shieldOR*\figwidth/2/\rpvOR/\sqrttwo);
    \draw[red,thick,->]  (0,2.3*2.54) node[above, anchor=south] {Core Barrel} --  (0,\coreBarrelOR/\rpvOR*\figwidth/2);
    \draw[red,thick,->]  (-2.3*2.54,0) node[left, anchor=east] {Baffle} --  (-\baffleOR/\rpvOR*\figwidth/2,0);
    \draw[red,thick,->]  (2.3*2.54,0) node[right, anchor=west] {RPV Liner} --  (\rpvIR/\rpvOR*\figwidth/2,0);
    
  \end{tikzpicture}
  
  \caption[Core cross-section]{Core cross-section indicating radial structures
  and enrichment loading pattern (cycle 1). Black denote stainless steel, dark
  gray denotes carbon steel, light blue denotes water, and red, yellow, and dark
  blue denote the 1.6, 2.4, and 3.1 w/o U235 regions, respectively.
  \label{fig_overview}}
\end{figure}



%%%%%%%%%%%%%%%%%%%%%%%%%%%%%%%%%%%%%%%%%%%%%%%%%%%%%%%%%%%%%%%%%%%%%%%%%%%%%%%%
\subsection{Radial Geometry}

  \input{./specifications/pin/pintypes}
  \input{./specifications/assy/assemblytypes}
  \input{./specifications/core/corespec}

%%%%%%%%%%%%%%%%%%%%%%%%%%%%%%%%%%%%%%%%%%%%%%%%%%%%%%%%%%%%%%%%%%%%%%%%%%%%%%%%  
\subsection{Axial Geometry}

  \input{./specifications/axial/axialspec}

%%%%%%%%%%%%%%%%%%%%%%%%%%%%%%%%%%%%%%%%%%%%%%%%%%%%%%%%%%%%%%%%%%%%%%%%%%%%%%%%
\subsection{Materials}

  \input{./specifications/materials/fuel_disc}
  \input{./specifications/materials/air_disc}
  \input{./specifications/materials/borosilicate_disc}
  \input{./specifications/materials/aic_rod_disc}
  \input{./specifications/materials/b4c_rod_disc}
  \input{./specifications/materials/helium_disc}
  \input{./specifications/materials/inconel_disc}
  \input{./specifications/materials/SS304_disc}
  \input{./specifications/materials/zirc_disc}
  \input{./specifications/materials/water_disc}
  \input{./specifications/materials/water_spn_disc}
  \input{./specifications/materials/ss_spn_disc}
  \input{./specifications/materials/carbonsteel_disc}

%\subsection{Notes on Simplifications}
%\label{sec:simplifications}

%\subsubsection{List of Simplifications}

%A list of the main simplifications employed in the previously described model is
%presented in Table \ref{table_simps}.

%\begin{table}[htpb]

%  \caption{Key simplifications. \label{table_simps}}
%  \begin{tabular}{l}
%    \\
%    \hline\hline
%    \\ undetailed support plate, nozzles, end plugs for fuel rods, BAs, and CRs
%    \\ guessed pin plenum spring radius
%    \\ guessed shield pad thickness
%    \\ guessed RPV thickness
%    \\ no startup rods (they replace BAs in some places)
%    \\ inaccurate lower and upper core plenums
%    \\ thermal pad positioning
%  \end{tabular}

%\end{table}

%\subsubsection{Radial Symmetry}

%The only source of radial asymmetry is from the instrument tubes. If ignored,
%octant symmetry might be used to cut down the size of the problem significantly.


%\subsubsection{Grid Spacer Simplifications}

%The important thing here is to conserve the mass, so if desired the stainless
%steel and inconel can be smeared together and distributed eveny accross the
%assembly as desired, either with a similar box geometry as presented here or as
%additional rings around each of the pins.  This could confine the construction
%of the axial geometry to only the pincells, simplifying the fuel assembly
%construction.
